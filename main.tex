\documentclass[11pt]{article}
\usepackage[utf8]{inputenc}
\RequirePackage[a4paper, inner=2.8cm ,outer=2.8cm, top=2.5cm, bottom=2.5cm, footskip=1cm]{geometry}
\usepackage{setspace}
\usepackage{hyperref}
\usepackage{fancyhdr}
\usepackage{lastpage}
\usepackage{bibentry}
\usepackage{siunitx}
\usepackage{longtable}
\sisetup{detect-weight=true, detect-family=true}
%\usepackage[ddmmyyyy]{datetime}

\pagestyle{fancy}
\renewcommand{\footrulewidth}{0pt}% default is 0pt
\renewcommand{\headrulewidth}{0pt}% default is 0pt
\fancyhf{}
%\lhead{Department of Engineering and Applied Sciences}
%\lhead{Application for Engineering and Applied Sciences Ph.D.}
%\rhead{Università degli Studi di Bergamo}
%\rhead{XXXVIII Cycle}
%\lfoot{Luca Ghislotti}
\cfoot{Page \thepage \hspace{1pt} of \pageref*{LastPage}}
%\rfoot{\date{\today}}

\title{\huge Scientific and Academic Curriculum%\\ Submitted for admission to Ph.D. in \\ ENGINEERING AND APPLIED SCIENCES \\ (XXXVIII Cycle)
}

\onehalfspacing
\author{\vspace{0.2cm}\begin{huge}Luca Ghislotti\end{huge}\\\vspace{0.1cm} Department of Engineering and Applied Sciences\\University of Bergamo}
\date{October 28, 2022}

\begin{document}
\maketitle
\thispagestyle{fancy}

\onehalfspacing

\section*{Personal Details}
\begin{tabbing}
\hspace{8pt}\textbf{Forename and Surname: } \= Luca Ghislotti\\
\hspace{6pt}\textbf{Place and Date of Birth: } \> Treviglio (BG), 03/02/1998\\
\hspace{76.5pt}\textbf{Nationality: } \>Italian\\
\hspace{94pt}\textbf{Address: } Via delle Crocette, SNC 24050 Ghisalba (BG)\\
\hspace{100.5pt}\textbf{Mobile: } +39 3331133825\\
\hspace{6.5pt}\textbf{Personal Email Address: } \href{mailto:ghislottiluca@gmail.com}{ghislottiluca@gmail.com}\\
\hspace{0pt}\textbf{Academic Email Address: } \href{mailto:l.ghislotti@studenti.unibg.it}{luca.ghislotti@unibg.it}\\
\hspace{105pt}\textbf{Skype: } \href{https://join.skype.com/invite/m1XvXqe82tbO}{Luca Ghislotti}\\
\hspace{96.5pt}\textbf{GitHub: } \href{https://github.com/lucaghislo}{github.com/lucaghislo}\\
\hspace{89.5pt}\textbf{LinkedIn: } \href{https://www.linkedin.com/in/luca-ghislotti/}{linkedin.com/in/luca-ghislotti}\\
\hspace{55.5pt}\textbf{Google Scholar: } \href{https://scholar.google.com/citations?user=Xt1\_bCYAAAAJ}{scholar.google.com}\\
\hspace{31.5pt}\textbf{Personal Web Page: } \href{https://lucaghislotti.com}{https://lucaghislotti.com}\\
\hspace{25pt}\textbf{Academic Web Page: } \href{https://didattica-rubrica.unibg.it/ugov/person/135728}{https://didattica-rubrica.unibg.it} 
\end{tabbing}

\bigskip
\section*{Academic Education and Training }

\subsection*{Doctor of Philosophy (Ph.D.) -  Microelectronics for High Energy Physics}
Università degli Studi di Bergamo, Bergamo (2022 - 2025)\\

\subsection*{Master of Science (M.Sc.) - Computer Engineering (LM-32)}
Università degli Studi di Bergamo, Bergamo (2020 - 2022)\\
\vspace{-10pt}
\begin{longtable}{r l}
    \renewcommand{\arraystretch}{1.5}
     \textbf{Thesis:} & ``Characterisation of the readout electronics of the Si(Li) tracker for the \\
     & first flight of the GAPS experiment" \\
     \textbf{Supervisor:} & Prof. Massimo Manghisoni \\
     \textbf{Keywords:} & GAPS, Dark Matter, Spectrometer, Si(Li), ASIC \\
     \textbf{Final Grade:} & 110/110 cum laude
\end{longtable}

\subsection*{Bachelor of Science (B.Sc.) - Computer Engineering (LM-8)}
Università degli Studi di Bergamo, Bergamo (2017 - 2020)\\
\vspace{-10pt}
\begin{longtable}{r l}
    \renewcommand{\arraystretch}{1.5}
     \textbf{Thesis:} & ``Data Anonymization Techniques: Implementation in the Apache Spark \\
     & Environment" \\
     \textbf{Supervisor:} & Prof. Stefano Paraboschi \\
     \textbf{Keywords:} & Data Anonymization, Apache Spark, Apache Hadoop, Privacy, Computer\\
     & Security \\
     \textbf{Final Grade:} & 107/110
\end{longtable}

\bigskip
\section*{Work Experience}
\subsection*{Intern at PRSE Srl}
\textit{Internship project regarding the development of a self-driving AV}\\
Bergamo, Italy\\
February 2022 - March 2022

\subsection*{Shareholder of Efficient Farm Engineering Srl}
\textit{Computer Engineer at an innovative startup operating in the agricultural sector}\\
Ghisalba (BG), Italy\\
December 2019 - Ongoing

\subsection*{Shareholder of Società Agricola Le Campagnole Srl}
\textit{Manager and IT Consultant of an organic farm with production and sale of dairy products}\\
Cologno al Serio (BG), Italy\\
January 2018 - October 2019

\bigskip
\section*{Scholarships and Certificates}
\subsection*{Ph.D. Scholarship}
Dottorato di Ricerca in INGEGNERIA E SCIENZE APPLICATE (XXXVIII ciclo)\\
Università degli Studi di Bergamo, Bergamo\\
\textit{Years:} 2022 - 2025\\
\subsection*{TOP 10 Student Program}
Università degli Studi di Bergamo, Bergamo\\
\textit{Years:} 2017 - 2018, 2020 - 2021, 2021 - 2022\\
\textit{Fee exemption award issued by Università degli Studi di Bergamo to best students}

\subsection*{Io e Lode - Studenti Eccellenti Scuole Bergamasche}
Confindustria Bergamo, Bergamo\\
\textit{Years:} 2013 - 2014, 2014 - 2015, 2015 - 2016, 2016 - 2017\\
\textit{Award issued by Confindustria Bergamo to best upper secondary school students}

\subsection*{FCE (B2 First)}
Cambridge Assessment English, Cambridge (2017)

\bigskip
\section*{Memberships}

\subsection*{INFN Memeber}
Istituto Nazionale di Fisica Nucleare, Pavia\\
Member of INFN CSN2 group as Technological Ph.D., section of Pavia.

\subsection*{IEEE Graduate Student Member}
Institute of Electrical and Electronics Engineers\\
Membership number \texttt{97046986}

\subsection*{ORCID Account}
ORCID ID \href{https://orcid.org/0000-0002-7084-5979}{\texttt{https://orcid.org/0000-0002-7084-5979}}

\bigskip
\section*{Language skills}
\paragraph{\hspace{6pt}Italian:}Mother tongue
\vspace{-0.5cm}
\paragraph{\hspace{2pt}English:} Fluent (C1)
\vspace{-0.5cm}
\paragraph{Spanish:} Intermediate (B2)

\bigskip
\section*{Computer skills}
\begin{tabbing}
\textbf{\hspace{16pt}Advanced knowledge: } \= C/C++, Java, Python, \LaTeX, MATLAB, \\ \>JavaScript, SQL, Scala, HTML, PHP, Linux, \\ \> GIT\\\\
\textbf{Intermediate knowledge: } LTSpice, B\&R Automation Studio (ST, SFC,\\ \> Ladder), MongoDB, Apache Spark, \\ \> REST API, Autodesk AutoCAD, Autodesk \\ \> Fusion \\\\
\textbf{\hspace{35pt}Basic Knowledge: } XPath, XQuery, Android Apps Development, \\ \>Assembly, Simulink, Autodesk Eagle
\end{tabbing}

\bigskip
\section*{Publications and Presentations Co-Authorships}

\noindent
V. RE, L. GHISLOTTI, P. LAZZARONI, M. MANGHISONI, E. RICEPUTI, L. RATTI, M. BOEZIO, G. ZAMPA, L. FABRIS, \textit{``A mixed-signal processor for X-ray spectrometry and tracking in the GAPS experiment"}, Nuclear Instruments and Methods in Physics Research Section A: Accelerators, Spectrometers, Detectors and Associated Equipment, \href{https://doi.org/10.1016/J.NIMA.2022.167617}{https://doi.org/10.1016/J.NIMA.2022.167617}. \\

\noindent
V. RE, L. GHISLOTTI, P. LAZZARONI, M. MANGHISONI, E. RICEPUTI, L. RATTI, M. BOEZIO, G. ZAMPA, L. FABRIS, \textit{``A 32-channels mixed-signal processor for the tracking system of the GAPS experiment"}, 15th Pisa Meeting on Advanced Detectors, La Biodola, Isola d'Elba, May 27, 2022, \href{https://agenda.infn.it/event/22092/contributions/167321/attachments/91411/124353/Re_Elba_2022.pdf}{https://agenda.infn.it/event/22092/contributions/167321}.\\

\noindent
E. RICEPUTI, M. BOEZIO, L. FABRIS, L. GHISLOTTI, P. LAZZARONI, M. MANGHISONI, L. RATTI, V. RE, G. ZAMBPA, \textit{``The 32 analog channels readout for the long flight GAPS balloon experiment tracking system"}, SIE 2022 - Pizzo (VV), 7/9 September 2022, \linebreak
\href{https://events.dimes.unical.it/sie2022/wp-content/uploads/sites/18/2022/08/Scheduling-oralposter-sessions.pdf}{https://events.dimes.unical.it/sie2022}.

\bigskip
\section*{Scientific activity}
The scientific activity and research interest of Luca Ghislotti fall mainly in the design of low-noise, low-power analogue front-end integrated circuit for semiconductor detectors readout in high energy physics and their characterisation.\\

\noindent
The research activity to date encompasses the following:
\begin{enumerate}
    \item \textbf{Characterisation of the readout electronics of the Si(Li) tracker of the GAPS experiment}\\The research work consists in the validation of the flight items of the Si(Li) tracker of the General AntiParticle Spectrometer (GAPS) experiment scheduled for late 2023 from the McMurdo Station in Antarctica. The characterisation work is aimed at the validation of the readout electronics of the tracker and its calibration, with great concern for ultra-low noise performance and high energy particles detection accuracy.
    \item \textbf{Design of a \SI{65}{\nano\meter} CMOS readout ASIC for the second flight of the Si(Li) tracker of the GAPS experiment}\\
    The research work is focused on the design of an improved version of the Application Specific Integrated Circuit (ASIC) currently employed for the readout of the Si(Li) detectors of the GAPS tracker by moving from the current \SI{180}{\nano\meter} CMOS technology in wich the integrated circuit has been designed, to a more scaled \SI{65}{\nano\meter} CMOS technology.  This new chip will be used during the second long duration balloon flight experiment that will take off from the McMurdo Station in Antarctica at the end of 2025.
\end{enumerate}

\bigskip
\section*{Interests and Activities}
Electronics, Linux, Hi-Fi,  Audio Power Amplifier Design, Film Photography, Aerospace

\bigskip
\section*{Volunteer Experience}
Contributor at Informatici Senza Frontiere ONLUS (2018 - Ongoing)

\vspace{2cm}
\noindent
Pursuant to art. 46 and 47 of Presidential Decree 445/2000, I declare that the information included in my CV is true, being aware of the possible application of Article 76 of the same article in the event of a false declaration.\\
Pursuant to the Legislative Decree (D.Lgs.) no. 196/2003 and the Regulation (UE) 2016/679, the undersigned declares to be well informed that his/her personal data being collected here will be treated, also in electronic form, exclusively for the scope of the procedure related to this declaration and authorizes the collection of personal data for the fulfilment of this procedure.\\

\bigskip
Bergamo, 28/10/2022
\vspace{1cm}
% firma

\end{document}
